
\newpage

\section{Problemas}

\begin{enumerate}
    

    \item Tenemos un conjunto de partículas, de modo que cada una de ellas se puede describir utilizando un espacio de estados de dimensión 2, y una base de dicho espacio de estados la constituyen los vectores $\{\ket{u_1}, \ket{u_2}\}$. Sabemos que un tercio de las partículas se encuentran en el estado $\ket{a} = \ket{u_1}$ y el resto en el estado $\ket{b} = \frac{1}{\sqrt{2}} (\ket{u_1} + i \ket{u_2})$.

    Encontrar la matriz densidad que describe al conjunto de partículas. \\
    Dado un observable:
    \[
    \hat{O} \equiv \begin{pmatrix} -7\alpha & -24 \alpha i \\ 24 \alpha i & 7\alpha \end{pmatrix},
    \]
    (siendo $\alpha$ una constante real) calcular: los valores que podemos obtener al medir el observable $\hat{O}$, las probabilidades correspondientes y el valor medio del observable. \textit{(24-01-14)}



    \item Tenemos un conjunto de partículas, de modo que cada una de ellas se puede describir utilizando un espacio de estados de dimensión 2, y una base de dicho espacio de estados la constituyen los vectores $\{\ket{u_1}, \ket{u_2}\}$. Sabemos que la mitad de las partículas se encuentran en el estado $\ket{a} = \ket{u_2}$ y el resto en el estado $\ket{b} = \frac{1}{\sqrt{5}} (\ket{u_1} + i2 \ket{u_2})$.

    Encontrar la matriz densidad que describe al conjunto de partículas. \\
    Dado un observable:
    \[
    \hat{O} \equiv \begin{pmatrix} 0 & -i \alpha \\ i \alpha & 0 \end{pmatrix},
    \]
    (siendo $\alpha$ una constante real) calcular: los valores que podemos obtener al medir el observable $\hat{O}$, las probabilidades correspondientes y el valor medio del observable. \textit{(09-04-14)}


    \item Tenemos un electrón descrito mediante el siguiente espinor:
    \[
    \ket{\psi} \equiv \begin{pmatrix} 2i e^{-r} \\ e^{-2r} \end{pmatrix}
    \]
    Normalizar el espinor. Calcular el valor medio de la coordenada radial, de $\hat{S}_z$ y de $\hat{S}_y$. \textit{(05-09-14)}


    \item Tenemos un conjunto de partículas, de modo que cada una de ellas se puede describir utilizando un espacio de estados de dimensión 2, y una base de dicho espacio de estados la constituyen los vectores $\{\ket{u_1}, \ket{u_2}\}$. Sabemos que la mitad de las partículas se encuentran en el estado $\ket{a} = \frac{1}{\sqrt{2}} (\ket{u_1} + \ket{u_2})$ y el resto en el estado $\ket{b} = \frac{1}{\sqrt{5}} (\ket{u_1} + i2 \ket{u_2})$.

    Encontrar la matriz densidad que describe al conjunto de partículas. \\
    Dado un observable:
    \[
    \hat{O} \equiv \begin{pmatrix} 0 & \alpha \\ \alpha & 0 \end{pmatrix},
    \]
    (siendo $\alpha$ una constante real) calcular: los valores que podemos obtener al medir el observable $\hat{O}$, las probabilidades correspondientes y el valor medio del observable. \textit{(15-12-14)}

    \item Tenemos un electrón descrito mediante el siguiente espinor:
    \[
    \langle \va{r}, \pm | \psi \rangle \equiv 
    \begin{pmatrix}
    \psi_{210}(\va{r}) + 2i \psi_{21-1}(\va{r}) + 3 \psi_{320}(\va{r}) \\
    (1 + i) \psi_{21-1}(\va{r}) + \psi_{310}(\va{r})
    \end{pmatrix}
    \]
    
    Normalizar el espinor. Calcular los valores que podemos obtener y los valores medios de los operadores \( \hat{H} \), \( \hat{L}^2 \), \( \hat{L}_z \), \( \hat{S}_z \) y \( \hat{S}_y \). \textit{(23-01-15)}

    \item Tenemos muchos sistemas que se pueden describir mediante un espacio de estados de dimensión 2, de modo que los vectores $\{\ket{u_1}, \ket{u_2}\}$ constituyen una base ortonormal de dicho espacio de estados. Sabemos que la mitad de los sistemas se encuentran en el estado $\ket{a} = (\ket{u_1} + i \ket{u_2}) / \sqrt{2}$, y la otra mitad en el estado $\ket{b} = (\ket{u_1} - \ket{u_2}) / \sqrt{2}$. Encontrar la matriz densidad que describe al conjunto de los sistemas. Si medimos el observable $\hat{O}$ que está representado por la siguiente matriz:
    \[
    \hat{O} = \alpha \begin{pmatrix} 9 & 2i \\ -2i & 6 \end{pmatrix}
    \]
    ¿qué valores podemos obtener? Utilizar la matriz densidad para calcular las probabilidades correspondientes. Calcular el valor medio de $\hat{O}$.  \textit{(01-09-16)}

    \item Tenemos un electrón descrito mediante el siguiente espinor:
    \[
    \langle \va{r}, \pm | \psi \rangle \equiv 
    \begin{pmatrix}
    \psi_{210}(\va{r}) + i \psi_{21-1}(\va{r}) + 3 \psi_{200}(\va{r}) \\
    (1 + 2i) \psi_{21-1}(\va{r}) + 3 \psi_{210}(\va{r})
    \end{pmatrix}
    \]
    
    Normalizar el espinor. ¿Está el electrón en un estado estacionario? ¿Por qué? Calcular los valores que podemos obtener y los valores medios de los operadores \( \hat{L}^2 \), \( \hat{L}_z \), \( \hat{S}_z \) y \( \hat{S}_y \). \textit{(12-01-16)}

    \item Tenemos muchos sistemas que se pueden describir mediante un espacio de estados de dimensión 2, de modo que los vectores $\{\ket{u_1}, \ket{u_2}\}$ constituyen una base ortonormal de dicho espacio de estados. Sabemos que un tercio de los sistemas se encuentran en el estado $\ket{a} = (\ket{u_1} + i \ket{u_2}) / \sqrt{2}$, un tercio en el estado $\ket{b} = \ket{u_1}$ y el resto en el estado $\ket{c} = \ket{u_3}$. 

    Encontrar la matriz densidad que describe al conjunto de los sistemas. Si medimos el observable $\hat{O}$ que está representado por la siguiente matriz:
    \[
    \hat{O} = \alpha \begin{pmatrix} 34 & -12i \\ 12i & 41 \end{pmatrix}
    \]
    ¿qué valores podemos obtener y con qué probabilidad? Calcular el valor medio de $\hat{O}$. \textit{(02-02-16)}

    \item Una partícula se encuentra descrita mediante el siguiente espinor:
    \[
    \ket{\psi} \equiv 
    \begin{pmatrix}
    e^{-r/2} \\
    (1 + i) e^{-r/2}
    \end{pmatrix}
    \]
    
    Normalizar el espinor. Calcular el valor medio de la coordenada radial. ¿Qué valores podemos obtener al medir \( \hat{S}_z \) y con qué probabilidad? ¿Y al medir \( \hat{S}_y \)? ¿Y al medir \( \hat{L}_z \)? \textit{(19-01-17)}
    
    \item Tenemos un conjunto formado por un gran número de sistemas, de modo que cada uno de ellos se puede describir mediante un espacio de estados de dimensión 2 y de modo que los vectores $\{\ket{u_1}, \ket{u_2}\}$ constituyen una base ortonormal de dicho espacio de estados. La mitad de los sistemas se encuentra en el estado $\ket{a} = (\ket{u_1} + \ket{u_2}) / \sqrt{2}$ y la otra mitad en el estado $\ket{b} = (\ket{u_1} - i \ket{u_2}) / \sqrt{2}$. 

    Encontrar la matriz densidad que describe al conjunto de los sistemas. El hamiltoniano de cada sistema está representado por la siguiente matriz:
    \[
    \hat{H} \equiv \alpha \begin{pmatrix} 9 & 2i \\ -2i & 6 \end{pmatrix}
    \]
    
    ¿Qué valores podemos obtener si medimos la energía de uno de los sistemas y con qué probabilidad podemos obtener cada uno de ellos? \textit{(09-02-17)}
    
    
    \item Tenemos un sistema que se puede describir mediante un espacio de estados de dimensión 2, de modo que los vectores $\{\ket{u_1}, \ket{u_2}\}$ constituyen una base ortonormal de dicho espacio de estados. El sistema se encuentra inicialmente en el estado:
    \[
    \ket{\psi (0)} = \frac{1}{\sqrt{2}} \ket{u_1} + \frac{i}{\sqrt{2}} \ket{u_2}
    \]
    
    El hamiltoniano del sistema está representado por la siguiente matriz:
    \[
    \hat{H} \equiv \begin{pmatrix} \hbar \omega & 0 \\ 0 & 2 \hbar \omega \end{pmatrix}
    \]
    
    El operador $\hat{O}$ está representado por la siguiente matriz:
    \[
    \hat{O} = \alpha \begin{pmatrix} 34 & -12i \\ 12i & 41 \end{pmatrix}
    \]
    
    Encontrar el operador $\hat{O}$ en la imagen de Heisenberg. Usar la expresión obtenida para calcular el valor medio de $\hat{O}$ en el instante \( t \). \textit{(15-01-19)}
    
    \item Tenemos una gran cantidad de sistemas que se pueden describir mediante un espacio de estados de dimensión 2, de modo que los vectores $\{\ket{u_1}, \ket{u_2}\}$ constituyen una base ortonormal de dicho espacio de estados. Sabemos que los sistemas están en una mezcla de los estados $\ket{a} = \ket{u_1}$ y $\ket{b} = (\ket{u_1} + i \ket{u_2}) / \sqrt{2}$. 

    Tenemos un dispositivo que permite medir el observable $\hat{O}$ que viene representado por la siguiente matriz:
    \[
    \hat{O} \equiv a \begin{pmatrix} 2 & 3i \\ -3i & 5 \end{pmatrix}
    \]
    siendo $a$ una constante real y positiva. ¿Cuál es el máximo y mínimo valor que podemos obtener para el valor medio de $\hat{O}$? \textit{(08-01-21)}

    \item Consideremos un sistema que se puede describir con un espacio de estados de dimensión 2 y una base de dicho espacio de estados la constituyen los vectores $\{\ket{u_1}, \ket{u_2}\}$. El operador hamiltoniano del sistema está representado en dicha base por la siguiente matriz:
    \[
    \hat{H} \equiv \hbar \omega \begin{pmatrix} 7 & 6i \\ -6i & -2 \end{pmatrix}
    \]
    
    Obtener el operador evolución del sistema en la base $\{\ket{u_1}, \ket{u_2}\}$. \textit{(09-02-21)}
    
    \item Tenemos dos partículas de espín $s_1 = s_2 = 1$, que se encuentran en el estado:
    \[
    \ket{2,1} + \ket{1,1} + \ket{1,0}
    \]
    
    Si medimos la componente $z$ del espín de la primera partícula, ¿qué valores podemos obtener y con qué probabilidad? Calcular la matriz densidad reducida de la primera partícula. \textit{(09-02-21)}

    \item Se considera el espacio de estados de espín de dos electrones $\mathcal{E} = \mathcal{E}_{S_1} \otimes \mathcal{E}_{S_2}$. En un instante dado se tiene que:
    \[
    \ket{\psi} = \frac{1}{\sqrt{6}} \left( i \ket{++} + 2 \ket{+-} + \ket{-+} \right)
    \]
    
    Se define la entropía de Von Neumann, como una extensión del concepto de entropía para un sistema cuántico y es una medida del "desconocimiento" que tenemos sobre el sistema, como:
    \[
    S = -\Tr (\hat{\rho} \ln \hat{\rho})
    \]
    
    De la misma forma podemos definir una entropía para el sistema 1 y otra para el sistema 2, considerando la matriz densidad reducida $\hat{\rho}_1$ y $\hat{\rho}_2$ respectivamente.
    \begin{enumerate}
        \item Calcular el valor máximo que puede tomar la entropía del sistema 1 ($S_{1\text{max}}$) y el valor mínimo ($S_{1\text{min}}$).
        \item Si definimos el grado de entrelazamiento del sistema 1 con el 2 de la forma:
        \[
        e = \frac{S_1 - S_{1\text{min}}}{S_{1\text{max}} - S_{1\text{min}}},
        \]
        calcular este valor para el estado que nos dan.
        \item Calcular el grado de entrelazamiento del sistema 2 con el 1 e indicar si se obtiene un resultado coherente.
    \end{enumerate} \textit{(21-01-22)}

    \item Se tiene un sistema que se puede describir mediante un espacio de estados de dimensión 3. El sistema se encuentra en el estado $\ket{\psi} = (\ket{u_1} + i \ket{u_3}) / \sqrt{2}$. El hamiltoniano está representado por el siguiente operador:
    \[
    \hat{H} = \hbar \omega \begin{pmatrix} 0 & i & -i \\ -i & 0 & 0 \\ i & 0 & 0 \end{pmatrix}
    \]
    siendo $\omega$ una constante real y positiva.
    
    Calcular:
    \begin{enumerate}
        \item[a)] Matriz densidad del sistema.
        \item[b)] Valores de energía del sistema (Autovalores).
        \item[c)] Vector correspondiente al estado fundamental, así como el proyector sobre dicho vector.
        \item[d)] Probabilidad de que al medir la energía obtengamos el mínimo valor posible.
        \item[e)] Valor medio de la energía utilizando el operador densidad.
    \end{enumerate} \textit{(11-02-22)}

    
    \item Considere el espinor:
    \[
    \langle \va{r}, \pm | \psi \rangle \equiv 
    \begin{pmatrix}
    e^{-r/2} \\
    (1 + i)e^{-r/2}
    \end{pmatrix}
    \]
    
    \begin{enumerate}
        \item[a)] Normalizar el espinor. ¿Qué valores podremos obtener al medir \( \hat{S}_z \) y con qué probabilidad?
        \item[b)] ¿Qué valores podremos obtener al medir \( \hat{S}_y \)? ¿Y \( \hat{L}_z \)?
        \item[c)] Obtener el valor medio de la coordenada radial.
    \end{enumerate}
    
    En \( t = 0 \) medimos el valor del spin en el eje \( z \) y obtenemos \( +\hbar/2 \). En ese mismo instante comienza a actuar sobre el sistema una perturbación con la forma:
    \[
    \hat{W} = \omega \hat{S}_x
    \]
    
    \begin{enumerate}
        \item[d)] Calcular el estado del sistema en el instante \( t \) usando teoría de perturbaciones hasta primer orden.
    \end{enumerate} \textit{(19-01-23)}


    \item Suponga que tenemos un sistema con momento angular 1. Se toma la base eligiendo los autovectores del momento angular \( J_z \) con autovalores \( +\hbar \), \( 0 \) y \( -\hbar \) respectivamente. El sistema está descrito por la siguiente matriz densidad:
    \[
    \rho = \frac{1}{4} \begin{pmatrix} 2 & 1 & 1 \\ 1 & 1 & 0 \\ 1 & 0 & 1 \end{pmatrix}
    \]
    
    \begin{enumerate}
        \item[a)] ¿Es \( \rho \) una matriz densidad válida? ¿Describe un estado puro o un estado mezcla? Razona tus respuestas.
        \item[b)] ¿Cuál es el valor medio de \( J_z \) en el sistema descrito por \( \rho \)?
        \item[c)] ¿Cuál es la desviación estándar de las medidas de \( J_z \)?
    \end{enumerate} \textit{(07-02-23)}
    
    
    
    

\end{enumerate}