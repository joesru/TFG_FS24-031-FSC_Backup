\newpage
\section{Problemas}

\begin{enumerate}
    
    \item Una partícula de masa \( m \) que está sometida al potencial de un oscilador armónico simple de frecuencia \( \omega \), se encuentra en \( t = 0 \) en el estado:
    \[
    \ket{\psi (0)} = \frac{1}{\sqrt{2}} \ket{\varphi_0} + \frac{i}{\sqrt{2}} \ket{\varphi_1}
    \]
    A partir de dicho instante, comienza a actuar una perturbación de la forma:
    \[
    \hat{W}(t) = \alpha \hat{x} \delta (t - \tau)
    \]
    Aplicar la teoría de perturbaciones hasta primer orden para calcular la probabilidad de que la partícula se encuentre en el estado \( \ket{\varphi_1} \) para \( t > \tau \). \textit{(24-01-14)}
    
    \item Una partícula de masa \( m \) que está sometida al potencial de un oscilador armónico simple de frecuencia \( \omega \), se encuentra en \( t = 0 \) en el estado \( \ket{\psi (0)} = \ket{\varphi_5} \). A partir de dicho instante, comienza a actuar una perturbación de la forma:
    \[
    \hat{W}(t) = \alpha \hat{x} e^{-t/\tau}
    \]
    Aplicar la teoría de perturbaciones hasta primer orden para calcular la probabilidad de que la partícula se encuentre en el estado \( \ket{\varphi_4} \) para \( t \to \infty \). \textit{(09-04-14)}

    \item Una partícula de masa \( m \) que está sometida al potencial de un oscilador armónico simple de frecuencia \( \omega \), se encuentra en \( t = 0 \) en el estado fundamental. A partir de dicho instante, comienza a actuar una perturbación de la forma:
    \[
    \hat{W}(t) = \alpha \hat{x}^2 e^{-t/\tau}
    \]
    Aplicar la teoría de perturbaciones hasta primer orden para calcular la probabilidad de que la partícula se encuentre en el segundo estado excitado para \( t \to \infty \). \textit{(05-09-14)}
    
    \item Una partícula de masa \( m \) que está sometida al potencial de un oscilador armónico simple de frecuencia \( \omega \), se encuentra en \( t = 0 \) en el estado \( \ket{\psi(0)} = \ket{\varphi_0} \). A partir de dicho instante, comienza a actuar una perturbación de la forma:
    \[
    \hat{W}(t) = \alpha \hat{x}^2 e^{-t/\tau}
    \]
    Aplicar la teoría de perturbaciones hasta primer orden para calcular la probabilidad de que la partícula se encuentre en el estado \( \ket{\varphi_2} \) para \( t \to \infty \). \textit{(15-12-14)}

    \item  Una partícula de masa \( m \) se encuentra en \( t = 0 \) en el estado fundamental de un pozo infinito de potencial de anchura \( a \), entre \( -a/2 \) y \( a/2 \). A partir de dicho instante, actúa una perturbación de la forma:
    \[
    \hat{W}(t) = \alpha_1 a \hat{x} e^{-t/\tau_1} + \alpha_2 \hat{x}^2 e^{-t/\tau_2}
    \]
    donde \( \alpha_1 \), \( \alpha_2 \), \( \tau_1 \) y \( \tau_2 \) son constantes reales. Aplicar la teoría de perturbaciones dependiente del tiempo para calcular la probabilidad de que la partícula se encuentre en el primer estado excitado y en el segundo estado excitado para \( t \to \infty \). \textit{(23-01-15)}

    \item Una partícula de masa \( m \) se encuentra en \( t = 0 \) en el primer estado excitado de un oscilador armónico simple de frecuencia \( \omega \). Si sometemos la partícula a una perturbación de la forma:
    \[
    \hat{W}(t) = \alpha \hat{p} e^{-t/\tau}
    \]
    Utilizar la teoría de perturbaciones dependiente del tiempo para calcular la probabilidad de que la partícula se encuentre en el estado fundamental, para tiempos muy cortos y para \( t \to \infty \). \textit{(12-01-16)}

    \item El hamiltoniano de una partícula es \( \hat{H} = \hat{L}_z \omega \) y se encuentra en \( t = 0 \) en el estado \( | l, m \rangle = | 1, 0 \rangle \). Si sometemos la partícula a una perturbación de la forma:
    \[
    \hat{W} = \alpha \hat{L}_x e^{-t/\tau}
    \]
    calcular la probabilidad de que la partícula se encuentre en el estado \( | 1, 1 \rangle \) para \( t \to \infty \), aplicando la teoría de perturbaciones dependiente del tiempo hasta primer orden. \textit{(02-02-16)}

    \item Una partícula está sometida al potencial de un oscilador armónico simple con frecuencia \( \omega \) y se encuentra en \( t = 0 \) en el estado fundamental. A partir de dicho instante, comienza a actuar una perturbación de la forma:
    \[
    \hat{W} = \alpha \left( a^3 + a t^3 \right) e^{-t/\tau}
    \]
    Calcular la probabilidad de que en \( t \to \infty \) la partícula se encuentre en el tercer estado excitado. \textit{(01-09-16)}

    \item Una partícula está sometida al potencial de un oscilador armónico simple con frecuencia \( \omega \) y se encuentra en \( t = 0 \) en el estado fundamental. A partir de dicho instante, comienza a actuar una perturbación de la forma:
    \[
    \hat{W}(t) = \alpha \hat{p} e^{-t/\tau}
    \]
    Calcular la probabilidad de transición al primer estado excitado para tiempos muy muy cortos y para tiempos muy largos. \textit{(19-01-17)}
    
    \item Una partícula está sometida al potencial de un oscilador armónico simple con frecuencia \( \omega \) y se encuentra en \( t = 0 \) en el estado fundamental. A partir de dicho instante, comienza a actuar una perturbación de la forma:
    \[
    \hat{W}(t) = \alpha \hat{p} e^{-t/\tau}
    \]
    Utilizar la teoría de perturbaciones dependiente del tiempo hasta primer orden para calcular el estado en el instante \( t \). ¿A qué tiende el estado para tiempos muy grandes? Calcular el valor medio de la posición para tiempos muy grandes. \textit{(09-02-17)}


    \item Una partícula se encuentra inicialmente en el segundo estado excitado de un oscilador armónico simple de frecuencia \(\omega\). Si sometemos la partícula a una perturbación de la forma:
    \[
    \hat{W}(t) = \alpha \hat{p}^2 e^{-t/\tau}
    \]
    aplicar la teoría de perturbaciones dependiente del tiempo hasta primer orden para encontrar las transiciones posibles para \(t \to \infty\), así como la probabilidad de transición correspondiente. \textit{(15-01-19)}

    \item Una partícula se encuentra inicialmente en el segundo estado excitado de un oscilador armónico simple de frecuencia \(\omega\). Si sometemos la partícula a una perturbación de la forma:
    \[
    \hat{W}(t) = \alpha \hat{x}^2 e^{-t/\tau}
    \]
    aplicar la teoría de perturbaciones dependiente del tiempo hasta primer orden para encontrar las transiciones posibles para \(t \to \infty\), así como la probabilidad de transición correspondiente. \textit{(08-01-19)}

    \item Tenemos una partícula inicialmente en el estado fundamental de un oscilador armónico simple de frecuencia \(\omega\). A partir del instante inicial, sometemos la partícula a la siguiente perturbación:
    \[
    \hat{W}(t) = \alpha e^{-t^2/\tau^2} \hat{x}
    \]
    Calcular, hasta primer orden en teoría de perturbaciones, la probabilidad de encontrar la partícula en el primer estado excitado para tiempos muy muy cortos. \textit{(09-02-21)}

    \item Una partícula de masa \(m\) está sometida al potencial de un oscilador armónico simple de frecuencia \(\omega\) y se encuentra inicialmente en el estado \(|\varphi_2\rangle\). A partir de dicho instante, se le aplica una perturbación de la siguiente forma:
    \[
    \hat{W}(t) = \alpha \left( \hat{a}^\dagger \hat{a}^\dagger \hat{a} + \hat{a}^\dagger \hat{a} \right) e^{-t/\tau}
    \]
    siendo \(\alpha\) una constante real. Calcular, hasta primer orden en teoría de perturbaciones:
    \begin{enumerate}
        \item Los estados en los que podemos encontrar la partícula en el instante \(t\).
        \item Los estados en los que podemos encontrar la partícula para \(t \to \infty\), así como las probabilidades correspondientes.
    \end{enumerate} \textit{(21-01-22)}

    \item Se tiene un oscilador armónico de masa \(m\) y frecuencia \(\omega\) inicialmente en el estado \(|\varphi_0\rangle\). A partir de dicho instante, se le aplica una perturbación al sistema de la siguiente manera:
    \[
    \hat{W}(t) = \alpha \hat{x} \hat{p} \hat{x} e^{-t/\tau}
    \]
    Calcular: 
    \begin{enumerate}
        \item Los estados en los que podemos encontrar al sistema en el instante \(t\)
        \item Los estados en los que podemos encontrar al sistema en \(t \to \infty\) así como las probabilidades correspondientes.
    \end{enumerate} \textit{(11-02-22)}

    
    

\end{enumerate}