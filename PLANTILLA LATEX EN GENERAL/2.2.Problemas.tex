\newpage

\section{Problemas}

\begin{enumerate}
    \item Tenemos dos bosones idénticos de espín \( s_1 = s_2 = 2 \), que interaccionan mediante el siguiente hamiltoniano:
    \[
    \hat{H} = \frac{\omega}{\hbar} \left( \hat{S}_1 + \hat{S}_2 \right)^2 + \omega \left( \hat{S}_{1z} + \hat{S}_{2z} \right)
    \]
    
    Encontrar los estados estacionarios en los que se pueden encontrar el conjunto de las dos partículas: su energía y su grado de degeneración.

    En \( t = 0 \) el sistema formado por las dos partículas se encuentra en el estado:
    \[
    \ket{\psi (0)} = \ket{2, 2; 1, 0} + \ket{2, 2; 0, 1}
    \]
    
    Encontrar el estado en el instante \( t \). Si en dicho instante medimos la componente del espín en la dirección del eje \( z \) de una de las partículas, ¿qué resultados podemos encontrar y con qué probabilidad? \textit{(24-01-14)}

    \item Tenemos dos partículas, que son fermiones idénticos de espín \( s_1 = s_2 = 3/2 \). Escribir los vectores de la forma \( \ket{J, M} \) en los que se pueden encontrar el conjunto de las dos partículas. Si las partículas se encuentran en el estado
    \[
    \ket{\psi} = \frac{2}{\sqrt{5}} \ket{0, 0} + \frac{i}{\sqrt{25}} \ket{2, 1}
    \]
    y medimos la componente \( z \) del espín de una de ellas, ¿qué resultados podemos obtener y con qué probabilidad? \textit{(09-04-14)}

    \item Tenemos dos partículas: una con espín \( s_1 = 2 \) y la otra con espín \( s_2 = 1 \). En el instante inicial se encuentran en el estado \( \ket{\psi (0)} = \ket{2, 1; 0, 1} \). El operador hamiltoniano del conjunto de las dos partículas viene dado por la siguiente expresión:
    \[
    \hat{H} = \omega \left( \hat{S}_{1z} + \hat{S}_{2z} \right)
    \]
    
    Calcular el estado en el instante \( t \). Si en dicho instante medimos la componente \( z \) del espín de la primera partícula, calcular la probabilidad de que obtengamos el valor \( 2\hbar \). \textit{(05-09-14)}

    \item Tenemos un sistema formado por dos partículas idénticas de espín \( s = 1 \). El operador hamiltoniano del sistema viene dado por el operador:
    \[
    \hat{H} = \omega \hat{S}_{1z} + \omega \hat{S}_{2z} - \frac{\omega}{\hbar} \hat{S}_1 \cdot \hat{S}_2
    \]
    
    Encontrar los valores que puede tomar la energía del sistema, su grado de degeneración y los estados estacionarios. Si el sistema se encuentra en el segundo estado excitado y realizamos una medida de la componente \( z \) del espín de las dos partículas, ¿qué resultados podemos obtener y con qué probabilidad? \textit{(15-12-14)}

    \item Tenemos dos partículas con espín \( s_1 = \frac{1}{2} \) y \( s_2 = 1 \), que se encuentran inicialmente con su momento angular acoplado en el estado \( \ket{\psi (0)} = \ket{\frac{1}{2}, -\frac{1}{2}} \). A partir de dicho instante se desacoplan, de modo que el hamiltoniano que describe la evolución temporal de las dos partículas es de la forma:
    \[
    \hat{H} = \omega \hat{S}_{1z}
    \]
    
    Calcular el estado de las dos partículas en el instante \( t \). Si en dicho instante medimos la componente \( z \) del espín de una de las dos partículas, ¿qué resultados podemos obtener y con qué probabilidad? Y si medimos el valor de \( \hat{J}^2 \), ¿qué valores podemos obtener y con qué probabilidad? \textit{(23-01-15)}
    
    
    \item Tenemos dos partículas idénticas de espín \( s_1 = s_2 = 2 \), cuyo hamiltoniano vale:
    \[
    \hat{H} = \frac{\omega}{\hbar} \hat{S}_1 \cdot \hat{S}_2 + \omega \left( \hat{S}_{1z} + \hat{S}_{2z} \right)
    \]
    
    Encontrar los estados estacionarios. Si el sistema se encuentra en el estado:
    \[
    \ket{\psi} = \frac{1}{\sqrt{2}} \ket{4, 1} + \frac{i}{\sqrt{2}} \ket{0, 0}
    \]
    
    y medimos la componente \( S_z \) de cada partícula, ¿qué resultados podemos obtener y con qué probabilidad? \textit{(12-01-16)}
    
    \item Tenemos dos partículas idénticas de espín \( s_1 = s_2 = 1 \), cuyo hamiltoniano vale:
    \[
    \hat{H} = \frac{\omega}{\hbar} \hat{S}_1 \cdot \hat{S}_2 + \omega \left( \hat{S}_{1z} + \hat{S}_{2z} \right)
    \]
    
    Encontrar los estados estacionarios, así como sus energías. El sistema se encuentra inicialmente en el estado:
    \[
    \ket{\psi (0)} = \ket{1, 1; 1, 0} + \ket{1, 1; 0, 1} + i \ket{1, 1; 1, 0, 0}
    \]
    
    Encontrar el estado en el instante \( t \). Si en dicho instante medimos el valor de \( \hat{J}^2 \), ¿qué resultados podemos obtener y con qué probabilidad? \textit{(02-02-16)}

    \item Tenemos dos partículas idénticas de espín \( s_1 = s_2 = 1 \), cuyo hamiltoniano vale:
    \[
    \hat{H} = \frac{\omega}{\hbar} \left( \hat{S}_1 + \hat{S}_2 \right)^2 + \omega \left( \hat{S}_{1z} + \hat{S}_{2z} \right)
    \]
    
    Encontrar los estados estacionarios, así como sus energías correspondientes. El sistema se encuentra inicialmente en el estado:
    \[
    \ket{\psi (0)} = \ket{1, 1; 1, 0} + \ket{1, 1; 0, 1} + 2i \ket{1, 1; 0, -1} + 2i \ket{1, 1; -1, 0}
    \]
    
    Encontrar el estado en el instante \( t \). Si en dicho instante medimos el valor de \( \hat{J}^2 \), ¿qué resultados podemos obtener y con qué probabilidad? \textit{(01-09-16)}

    \item Tenemos dos partículas idénticas de espín \( s_1 = s_2 = 3/2 \), cuyo hamiltoniano vale:
    \[
    \hat{H} = \omega \left( \hat{S}_{1z} + \hat{S}_{2z} \right)
    \]
    Encontrar los estados estacionarios, así como sus energías correspondientes. El sistema se encuentra inicialmente en el estado:
    \[
    \ket{\psi(0)} = \ket{2,1}
    \]
    Encontrar el estado en el instante \( t \). Si en dicho instante medimos el valor de \( \hat{S}_z \) de una de las partículas, ¿qué valores podemos obtener y con qué probabilidad? \textit{(19-01-17)}

    
    \item Tenemos un sistema formado por dos partículas idénticas de espín \( s = 1 \). El operador hamiltoniano del sistema viene dado por el operador:
    \[
    \hat{H} = \omega \hat{S}_{1z} + \omega \hat{S}_{2z} - \frac{\omega}{\hbar} \hat{S}_1 \cdot \hat{S}_2
    \]
    Encontrar los valores que puede tomar la energía del sistema, su grado de degeneración y los estados estacionarios. Si el sistema se encuentra en el primer estado excitado y realizamos una medida de la componente \( z \) del espín de las dos partículas, ¿qué resultados podemos obtener y con qué probabilidad? \textit{(09-02-17)}
    
    \item Tenemos dos partículas idénticas de espín \( s_1 = s_2 = 1 \), cuyo hamiltoniano vale:
    \[
    \hat{H} = \frac{\omega}{\hbar} \hat{J}_+ \hat{J}_-
    \]
    Encontrar los estados estacionarios, así como sus energías. El sistema se encuentra inicialmente en el estado:
    \[
    \ket{\psi(0)} = \hat{S} \left( \ket{1,1;1,0} + \ket{1,1;0,0} + i \ket{1,1;-1,0} \right)
    \]
    siendo \( \hat{S} \) el simetrizador. Encontrar el estado en el instante \( t \). Si en dicho instante medimos el valor de \( \hat{S}_z \) de una de las partículas, ¿qué resultados podemos obtener y con qué probabilidad? \textit{(15-01-19)}

    \item Tenemos dos partículas idénticas de espín \( s_1 = s_2 = 1 \), de modo que una de ellas se encuentra en el estado individual \( \ket{1,1} \) y la otra en el estado individual \( \left( \ket{1,0} + \ket{1,-1} \right)/\sqrt{2} \). Simetrizar convenientemente y escribir el estado en el que se encuentran las partículas. Si medimos el valor de \( S_z \) de una de las partículas, ¿qué resultados podemos obtener y con qué probabilidad? ¿Qué valores se pueden obtener al medir \( J^2 \) y con qué probabilidad? \textit{(08-01-21)}

    \item Tenemos dos partículas de espín \( s_1 = s_2 = 1 \), que se encuentran en el estado:
    \[
    \ket{2,1} + \ket{1,1} + \ket{1,0}
    \]
    Si medimos la componente \( z \) del espín de la primera partícula, ¿qué valores podemos obtener y con qué probabilidad? Calcular la matriz densidad reducida de la primera partícula. \textit{(09-02-21)}

    \item Se considera el espacio de estados de espín de dos partículas \( \mathcal{E} = \mathcal{E}_{S_1} \otimes \mathcal{E}_{S_2} \), donde los dos espines tienen los valores \( s_1 = 1 \) y \( s_2 = 2 \). Se define el hamiltoniano de interacción entre los dos espines de la forma:
    \[
    \hat{H} = \frac{\omega}{\hbar} \hat{S}_1 \cdot \hat{S}_2 - \omega (\hat{S}_{1z} + \hat{S}_{2z})
    \]
    Calcular los posibles valores de la energía y los autovectores correspondientes. Comprobar si existen valores degenerados e indicar uno de ellos así como su grado de degeneración.
    
    Si en el instante inicial el estado del sistema es:
    \[
    \ket{\psi (0)} = \ket{1, 2; 0, -2}
    \]
    Calcular el estado del sistema en el instante \( t \). Si medimos el valor de \( \hat{S}_{1z} \) en dicho instante, calcular los valores que podemos encontrar y la probabilidad correspondiente. \textit{(21-01-22)}

    \item \begin{itemize}
        \item[a)] Aplicar el método estándar para obtener los coeficientes de Clebsch-Gordan correspondientes al acoplo entre dos momentos angulares \( j = 1/2 \), que permiten escribir los vectores \( \ket{J, M} \) en función de los vectores \( \ket{j_1, j_2; m_1, m_2} \).
        \item[b)] Obtener los vectores \( \ket{J, M_y} \), autovectores de \( \hat{J}^2 \) y \( \hat{J}_y \), en función de los vectores \( \ket{j_1, j_2; m_1, m_2} \). 
    \end{itemize} \textit{(11-02-22)}

    \item Dos partículas idénticas de espín \( s_1 = s_2 = 1 \) se encuentran en un estado en el que la proyección del spin en el eje \( x \) de una de ellas vale \( \hbar \) y la proyección del espín de la otra en el mismo eje vale \( -\hbar \)

    \begin{itemize}
        \item[a)] Escribe el estado \( \ket{\psi} \) en el que se encuentra el sistema. Puedes usar el sistema de referencia que prefieras.
        \item[b)] ¿Qué probabilidades hay de medir la proyección del momento total \( M \) en el eje \( x \) y obtener 0?
    \end{itemize}
    
    El Hamiltoniano de este sistema se puede escribir como:
    \[
    \hat{H} = \frac{\omega}{\hbar} \left( \hat{S}_1 + \hat{S}_2 \right)^2 + \omega \left( \hat{S}_{1x} + \hat{S}_{2x} \right)
    \]
    
    \begin{itemize}
        \item[c)] ¿En qué estado se encontrará el sistema en el instante \( t \)?
    \end{itemize} \textit{(19-01-23)}

    \item Tenemos dos partículas de espín \( s_1 = s_2 = 1/2 \) cuyo Hamiltoniano vale:
    \[
    \hat{H} = \frac{2 \omega}{\hbar} \hat{s}_1 \cdot \hat{s}_2 + \omega \left( \hat{s}_{1z} + \hat{s}_{2z} \right)
    \]
    \begin{itemize}
        \item[a)] Encontrar los estados estacionarios, sus energías y el estado fundamental.
        \item[b)] El estado del sistema se puede describir, en \( t = 0 \), como:
        \[
        \ket{\psi(t = 0)} = \ket{1/2, 1/2; 1/2, 1/2}
        \]
        En ese mismo instante empieza a actuar una perturbación con la forma:
        \[
        \hat{W} = \omega \hat{S}_y
        \]
        Encontrar \( \ket{\psi(t)} \) usando teoría de perturbaciones hasta primer orden.
    \end{itemize} \textit{(07-02-23)}

    \item Suponga que tenemos un sistema con momento angular 1. Se toma la base eligiendo los autovectores del momento angular \( J_z \) con autovalores \( +\hbar \), \( 0 \) y \( -\hbar \) respectivamente. El sistema está descrito por la siguiente matriz densidad:
    \[
    \rho = \frac{1}{4} \begin{pmatrix} 2 & 1 & 1 \\ 1 & 1 & 0 \\ 1 & 0 & 1 \end{pmatrix}
    \]
    \begin{itemize}
        \item[a)] ¿Es \( \rho \) una matriz densidad válida? ¿Describe un estado puro o un estado mezcla? Razona tus respuestas.
        \item[b)] ¿Cuál es el valor medio de \( J_z \) en el sistema descrito por \( \rho \)?
        \item[c)] ¿Cuál es la desviación estándar de las medidas de \( J_z \)?
    \end{itemize} \textit{(07-02-23)} 
    
    

\end{enumerate}