\section{Cuestiones}

\subsection*{Momento angular total}

\begin{enumerate}
    
    \item Demostrar que el producto escalar \( \hat{L} \cdot \hat{S} \) conmutan con el momento angular total \( \hat{J} = \hat{L} + \hat{S} \). ¿Cómo se puede interpretar este resultado? \textit{(09-04-14)}
    
    \item Para un espacio de estados de dos espines $\mathcal{E} = \mathcal{E}_{S_1} \otimes \mathcal{E}_{S_2}$, demostrar que el producto escalar de los dos espines no conmute con $\hat{S}_{1x}$ ni con $\hat{S}_{2x}$, pero que sí conmute con $\hat{J}_x = \hat{S}_{1x} + \hat{S}_{2x}$. \textit{(01-09-16)}
    

\end{enumerate}

\subsection*{Composición de momentos angulares}

\begin{enumerate}
    
    \item Dada la composición de momentos angulares \( j_1 = 2 \) y \( j_2 = 3 \), aplicar el operador \( \hat{J}_- \) para obtener los vectores \( \ket{5,4} \) y \( \ket{5,3} \). \textit{(15-12-14)}
    

    \item En la composición de momentos angulares \( j_1 = 3, j_2 = 2 \), ¿cuántos vectores de la forma \( \ket{J, M} \) hay? ¿Para qué valores de \( J \) se produce un cambio de signo en los coeficientes de Clebsch-Gordan al intercambiar el papel que juegan \( j_1 \) y \( j_2 \)? \textit{(23-01-15)}

    \item Para la composición de momentos angulares \( j_1 = 2, j_2 = 1 \), el vector \( \ket{J, M} = \ket{1,1} \) se puede escribir en función de vectores de la base \( \ket{j_1, j_2; m_1, m_2} \) de la siguiente forma:
    \[
    \ket{1,1} = \sqrt{\frac{3}{5}} \ket{2, 1; 2, -1} - \sqrt{\frac{3}{10}} \ket{2, 1; 1, 0} + \frac{1}{\sqrt{10}} \ket{2, 1; 0, 1}
    \]
    
    Aplicar el operador \( \hat{J}_- \) a este vector para obtener el vector \( \ket{1,0} \). ¿Qué coeficientes de Clebsch-Gordan podemos deducir a partir de la expresión obtenida? \textit{(11-06-15)}

    \item Para la composición de momentos angulares $j_1 = 3, j_2 = 3$, calcular cuántos vectores de la forma $\ket{J, M}$ tendremos. ¿Cuáles son los posibles valores de $J$? ¿Qué vectores $\ket{J, M}$ serán simétricos ante el intercambio del papel que juegan las dos partículas y qué vectores serán antisimétricos? \textit{(12-01-16)}

    \item En la composición de dos momentos angulares $j_1 = 3$ y $j_2 = 2$, aplicar el operador $\hat{J}_-$ para obtener los primeros coeficientes de Clebsch-Gordan. ¿Cuáles son y cuánto valen? \textit{(02-02-16)}
    
    \item Dada la composición de momentos angulares $j_1 = 3, j_2 = 2$, aplicar el operador $\hat{J}_-$ de forma adecuada para obtener los vectores $\ket{5,4}$ y $\ket{5,3}$. \textit{(06-02-17)}

    \item Para la composición de momentos angulares \( j_1 = 3, j_2 = 1 \), ¿cuál será el vector \(| 4, -4 \rangle\) en función de los vectores \(| 3, 1; m_1, m_2 \rangle\)? Calcular el vector \(| 4, -3 \rangle\) a partir del anterior. \textit{(11-02-22)}   
    

    \item En la composición de momentos angulares $j_1 = 4$, $j_2 = 2$, calcular cuántos vectores $|J, M \rangle$ hay y calcular el vector $|6, 5\rangle$ en función de los vectores $|4, 2; m_1, m_2 \rangle$. \textit{(19-01-23)}
    
    \item Para obtener los vectores de la base $|J, M \rangle$ en función de los vectores $|j_1, j_2; m_1, m_2 \rangle$ hemos estudiado un método sistemático que consiste en partir del vector $|J, J \rangle = |j_1 + j_2, j_1 + j_2 \rangle = |j_1, j_2; j_1, j_2 \rangle$ y aplicar sucesivamente el operador $\hat{J}_-$ para encontrar todos los vectores con $J = j_1 + j_2$. A continuación se obtenía el vector $|j_1 + j_2 - 1, j_1 + j_2 - 1 \rangle$ y se aplicaba de nuevo sucesivamente el operador $\hat{J}_- \ldots$ Aplicar este método para la composición de momentos angulares $j_1 = j_2 = \frac{1}{2}$. \textit{(07-02-23)}

    \item Para la composición de momentos angulares $j_1 = 4$, $j_2 = 3$, ¿cuál es el vector $|J, M \rangle$ que tiene máximo valor de $J$ y de $M$? A partir de dicho vector obtener el vector $|J, M - 1\rangle$. ¿Qué coeficientes de Clebsch-Gordan podemos deducir? \textit{(05-09-23)}

    \item Para la composición de momentos angulares $j_1 = 2$, $j_2 = 2$, escribir el vector $|2, 1\rangle$ en función de los vectores en la que los momentos angulares individuales son diagonales. ¿Tiene este vector un carácter bien definido ante el intercambio de los dos momentos angulares? ¿Cuál? ¿Hay algún motivo por el que sea así? \textit{(07-09-23)}

    \item Para la composición de momentos angulares $j_1 = 3$, $j_2 = 3$, ¿cuántos vectores $|J, M \rangle$ hay? ¿Cuáles de ellos serán simétricos ante el intercambio de las dos partículas y cuáles antisimétricos? Obtener los coeficientes de Clebsch-Gordan correspondientes al vector $|6, 5 \rangle$. \textit{(11-01-24)}


\end{enumerate}

\subsection*{Coeficientes de Clebsch-Gordan}

\begin{enumerate}
    
    \item ¿Qué condiciones deben cumplir los coeficientes de C-G para que sean distintos de cero? \textit{(24-01-14)}
    
    \item Definir los coeficientes de Clebsch-Gordan. \textit{(05-09-14)}
    
    \item Indicar de cuáles de los siguientes coeficientes de Clebsch-Gordan se puede afirmar que son nulos e indicar por qué.
        \begin{itemize}
            \item \( \langle 2, 1; 1, 0 | 1, 1 \rangle \)
            \item \( \langle 3, 1; 2, 2 | 2, 4 \rangle \)
            \item \( \langle 2, -1; 1, 0 | 0, 1 \rangle \)
            \item \( \langle 4, 1; 3, 1 | 6, 4 \rangle \)
            \item \( \langle 2, 3; 1, 0 | 4, 2 \rangle \)
        \end{itemize} \textit{(05-09-14)}

        \item Para la composición de momentos angulares \( j_1 = 2, j_2 = 3/2 \), demostrar que los primeros coeficientes de Clebsch-Gordan coinciden con los de la tabla:
        \[
        \langle 2, \frac{3}{2}; 1, \frac{3}{2} | \frac{7}{2}, \frac{7}{2} \rangle = 1, \quad \langle 2, \frac{3}{2}; 2, \frac{1}{2} | \frac{7}{2}, \frac{5}{2} \rangle = \sqrt{\frac{4}{7}}
        \] \textit{(23-01-15)}

        \item Indicar de forma razonada qué coeficientes de Clebsch-Gordan son nulos y cuáles son distintos de cero:
        \begin{itemize}
            \item \(\langle 3, 1; 1, 1 | 4, 2 \rangle\)
            \item \(\langle 3, 2; 2, 1 | 2, 3 \rangle\)
            \item \(\langle 2, 3; 2, 2 | 6, 4 \rangle\)
            \item \(\langle 2, 2; 0, 0 | 3, 0 \rangle\)
        \end{itemize} \textit{(09-02-21)}
    
        \item Indicar de forma razonada qué coeficientes de Clebsch-Gordan son nulos y cuáles no:
        \begin{itemize}
            \item \(\langle 3, 1 | 2, 1; 2, 0 \rangle\)
            \item \(\langle 4, 2 | 2, 1; -1, 1 \rangle\)
            \item \(\langle 2, 3 | 1, 1; 0, 0 \rangle\)
            \item \(\langle 3, 2 | 2, 1; -1, 0 \rangle\)
            \item \(\langle 3, 3 | 2, 1; 1, 2 \rangle\)
        \end{itemize} \textit{(21-01-22)}

        \item Los siguientes coeficientes de Clebsch-Gordan son nulos. Explicar el motivo. Si hay más de un motivo explicar cada uno.
        \begin{itemize}
            \item $\langle 2, 1; 1, 1 | 1, 2 \rangle$
            \item $\langle 0, 2; 0, 1 | 1, 1 \rangle$
            \item $\langle 1, 1; 0, -1 | 1, 1 \rangle$
            \item $\langle 2, 2; 1, 2 | 3, 2 \rangle$
            \item $\langle 3, 1; 2, 2 | 0, 3 \rangle$
        \end{itemize} \textit{(02-02-24)}

\end{enumerate}

\subsection*{Propiedades de los coeficientes C-G}


\begin{enumerate}
    
    \item A partir de las relaciones de recurrencia de los coeficientes de Clebsch-Gordan, demostrar la siguiente expresión:
    \[
    \langle j, j; m, -m | 0, 0 \rangle = \frac{(-1)^{j-m}}{\sqrt{2j+1}}
    \]  \textit{(15-01-19)}

    \item Demostrar la primera relación de recurrencia de los coeficientes de Clebsch-Gordan. \textit{(08-01-21)}

\end{enumerate}

\subsection*{Operadores escalares, vectoriales y tensoriales}

\begin{enumerate}
    
    \item ¿Qué se entiende como operador escalar? \textit{(24-01-14)}     

    \item ¿Qué se entiende por operador vectorial? Demostrar que si el vector \( \ket{\alpha, J, M} \) es un autovector de \( \hat{J}_z \) con autovalor \( \hbar M \), el vector \( \hat{V}_z \ket{\alpha, J, M} \) (donde \( \hat{V}_z \) es la componente en la dirección \( z \) de un operador vectorial) también lo es y con el mismo autovalor. \textit{(15-12-14)}      
    
    
    \item ¿Qué se entiende por operador escalar? Demostrar que los elementos de matriz de un operador escalar $\bra{k', J', M'} \hat{A} \ket{k, J, M}$ son nulos, a menos que $J = J'$ y $M = M'$. \textit{(19-01-17)}     

\end{enumerate}