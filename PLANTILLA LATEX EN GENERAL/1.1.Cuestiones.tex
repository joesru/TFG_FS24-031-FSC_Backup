\section{Cuestiones}

\subsection*{Reducción del paquete de ondas}

\begin{enumerate}
    
    \item Tenemos una partícula en el estado $\ket{\psi} = \frac{1}{\sqrt{2}} (\ket{u_1} + i \ket{u_2})$, donde los vectores $\{\ket{u_1}, \ket{u_2}\}$ constituyen una base ortonormal del espacio de estados. Queremos medir un observable $\hat{O}$, del que sabemos que uno de los autovalores es $o$ y el proyector que proyecta sobre el subespacio donde se encuentran los autovectores de $\hat{O}$ con dicho autovalor vale:
    \[
    \hat{P}_o = \frac{1}{6} \begin{pmatrix} 5 & 1 + 2i \\ 1 - 2i & 1 \end{pmatrix}
    \]
    Calcular la probabilidad de que si medimos el observable $\hat{O}$ obtengamos como resultado de la medida el valor $o$. Si medimos $\hat{O}$ y obtenemos el valor $o$, ¿cuál será el estado justo después de la medida? \textit{(12-01-16)}

\end{enumerate}

\subsection*{Evolución temporal de un sistema}

\begin{enumerate}

    \item Definir las frecuencias de Bohr de un sistema e indicar cómo depende de ellas el valor medio de un observable, $\langle \hat{A} \rangle (t)$, que no depende explícitamente del tiempo. \textit{(07-02-23)}
    

\end{enumerate}

\subsection*{Imagen de Schrödinger, de Heisenberg y de interacción}

\begin{enumerate}
    
    \item Explica la diferencia entre trabajar en la imagen de Schrödinger y la de Heisenberg. \textit{(24-01-14)}
    
    \item Enumerar las propiedades del operador evolución. \textit{(09-04-14)}
    
    \item Obtener la ecuación diferencial que satisface el operador evolución $U (t,0)$. \textit{(05-09-14)}
    
    \item Obtener la ecuación que nos da la evolución temporal de un operador en la imagen de interacción. \textit{(15-12-14)}
    
    \item El hamiltoniano de una partícula es $\hat{H} = \hat{H}_0 + \hat{W}$, siendo todos estos operadores constantes. Si en $t = 0$ la partícula se encuentra en un estado $\ket{\varphi}$, que es autovector de $\hat{H}_0$ y $\hat{W}$, de modo que $\hat{H}_0 \ket{\varphi} = E_0 \ket{\varphi}$ y $\hat{W} \ket{\varphi} = E_1 \ket{\varphi}$, escribir el estado de la partícula en la imagen de Heisenberg y en la de interacción. \textit{(23-01-15)}
    
    \item Deducir las ecuaciones que nos dan la evolución temporal de un estado y de un operador en la imagen de interacción. \textit{(12-01-16)}
    
    \item Obtener ecuación que nos da la evolución temporal de un operador en la imagen de interacción. \textit{(19-01-17)}
    
    \item ¿Cuál es la expresión del operador $\hat{U}(t, t_0)$ para un hamiltoniano que no depende explícitamente del tiempo? Utilizar dicha expresión para demostrar que se verifican las propiedades del operador evolución. \textit{(09-02-17)}
    
    \item Escribir el operador evolución $\hat{U}(t, 0)$ para una partícula libre. A partir de la definición de $\hat{x}_H(t)$, utilizar la regla de conmutación de $\hat{U}(t, 0)$ con el operador $\hat{x}$ y obtener la evolución temporal de $\hat{x}_H(t)$. ¿Es razonable el resultado obtenido? \textit{(15-01-19)}
    
    \item El operador hamiltoniano que determina la evolución temporal de un sistema depende explícitamente del tiempo de la forma:
    \[
    \hat{H} = \hat{A} t^2
    \]
    siendo $\hat{A}$ un operador hermítico. Encontrar el operador evolución $\hat{U}(t, t_0)$. \textit{(08-01-21)}

    \item Un sistema se encuentra inicialmente en un estado $\ket{\varphi}$, de modo que $\ket{\varphi}$ es autovector común de un operador $\hat{H}_0$, con autovalor $E_0$, y de un operador $\hat{W}$, con autovalor $E_1$. Escribir el estado en el instante $t$ en la imagen de Schrödinger, en la de Heisenberg y en la de interacción, si el hamiltoniano es $\hat{H} = \hat{H}_0 + \hat{W}$. Comprobar que los estados anteriores satisfacen las ecuaciones de Schrödinger correspondientes a cada una de las imágenes. \textit{(21-01-22)}
    
    \item El estado de un sistema en el instante $t$ en la imagen de Schrödinger viene dado por $\ket{\psi(t)}$. Sabemos que una forma distinta de trabajar consiste en definir la imagen de Heisenberg en la que el estado no evoluciona y son los operadores que representan a las distintas magnitudes físicas los que evolucionan. ¿Cómo se puede definir un operador densidad en la imagen de Heisenberg? ¿Cómo se utilizaría dicho operador para calcular valores medios y probabilidades? \textit{(11-02-22)}
    
    \item Demostrar que el siguiente operador satisface todas las propiedades de un operador evolución:
    \[
    \hat{U}(t, t_0) = \begin{pmatrix} \cos \omega (t - t_0) & i \sin \omega (t - t_0) \\ i \sin \omega (t - t_0) & \cos \omega (t - t_0) \end{pmatrix}
    \]
    ¿Cómo se podría obtener el hamiltoniano del sistema? \textit{(19-01-23)}

    \item Demostrar la siguiente expresión:
    \[
    \frac{d\hat{A}_I}{dt} = \frac{1}{i\hbar} \left[ \hat{A}_S, \hat{H}_0 \right]_I + \left( \frac{d\hat{A}_S}{dt} \right)_I
    \] \textit{(07-02-23)}

    \item El estado de una partícula en $t = 0$ en la base $\{\ket{u_1}, \ket{u_2}\}$ es:
    \[
    \ket{\psi(0)} = \frac{1}{\sqrt{3}} \ket{u_1} + \sqrt{\frac{2}{3}} \ket{u_2}
    \]
    Si el hamiltoniano es $\hat{H} = \hat{H}_0 + \hat{W}(t)$ con:
    \[
    \hat{H}_0 \equiv \begin{pmatrix} \hbar \omega & 0 \\ 0 & 3\hbar \omega \end{pmatrix} \quad \text{y} \quad \hat{W}(t) \equiv \begin{pmatrix} 2\hbar \omega^2 t & 0 \\ 0 & 5\hbar \omega^2 t \end{pmatrix}
    \]
    encontrar el estado en el instante $t$ en la imagen de Heisenberg y en la de interacción. \textit{(05-09-2023)}

    \item El hamiltoniano de un sistema es:
    \[
    \hat{H} = \hat{A}t
    \]
    siendo $\hat{A}$ un operador hermítico. Obtener el operador evolución $\hat{U}(t, 0)$. Si $\ket{\psi(0)}$ es autovector de $\hat{A}$ con autovalor $a$, utilizar el operador evolución para obtener el estado en el instante $t$. \textit{(07-09-2023)}    

    \item Demostrar la ecuación que nos da la evolución temporal del operador evolución $\hat{U}(t, t_0)$. Si tenemos un hamiltoniano que depende del tiempo de la forma:
    \[
    \hat{H} = \hat{A} t^2
    \]
    donde $\hat{A}$ es un operador que no depende del tiempo, obtener el operador evolución $\hat{U}(t_1, t_2)$. \textit{(11-01-24)}

    \item Tenemos un operador en la imagen de Schrödinger $\hat{A} = \hat{A}_S$, que no depende explícitamente del tiempo. Demostrar que la evolución temporal del operador en la imagen de Heisenberg $\hat{A}_H (t)$ viene dada por:
    \[
    \frac{d\hat{A}_H(t)}{dt} = \frac{1}{i \hbar} \left[ \hat{A}_H(t), \hat{H}_H \right]
    \]
    Aplicar esta ecuación a los operadores $\hat{x}$ y $\hat{p}$ de una partícula libre que se mueve en una sola dimensión. Resolver las ecuaciones para obtener $\hat{x}_H(t)$ y $\hat{p}_H(t)$ y, finalmente, expresar estos dos últimos operadores en la representación coordenadas. \textit{(02-02-24)}

\end{enumerate}

\subsection*{Matriz densidad}

\begin{enumerate}
    
    \item ¿Qué necesidad hay de introducir el formalismo de la matriz densidad si se puede trabajar con los vectores de estado? \textit{(24-01-14)}
    
    \item Obtener la ecuación que nos da la evolución temporal de la matriz de propagación para el caso de un estado puro. \textit{(09-04-14)}
    
    \item Indicar si la siguiente matriz densidad corresponde a un estado puro o a uno mezcla: $$\hat{\rho} = \pmqty{ 1/2 & 1 + i \\ 1 - i & 1/2 }$$ \textit{(05-09-14)}

    \item Deducir si la siguiente matriz verifica las propiedades de una matriz densidad y si se trata de la correspondiente a un estado simple o uno mezcla:
    \[
    \hat{\rho} \equiv \frac{1}{3} \begin{pmatrix} 1 & -1 - i \\ -1 + i & 2 \end{pmatrix}
    \] \textit{(11-06-15)}

    \item Obtener la ecuación que determina la evolución temporal de la matriz densidad para el caso de un estado mezcla. \textit{(01-09-16)}

    \item Un conjunto de partículas se describe mediante la siguiente matriz densidad:
    \[
    \hat{\rho} = \frac{1}{3} \begin{pmatrix} 1 & 1 - i \\ 1 + i & 2 \end{pmatrix}
    \]
    Comprobar si verifica las propiedades de una matriz densidad e indicar si se trata de una matriz que describe un estado simple o mezcla. \textit{(12-01-16)}

    \item Un sistema que se puede describir mediante un espacio de estados de dimensión $2$, está descrito por la matriz densidad:
    \[
    \hat{\rho} = \frac{1}{4} \begin{pmatrix} 3 & -i \\ i & 1 \end{pmatrix}
    \]
    ¿Se trata de un estado puro o mezcla? Dado el observable $\hat{O} = \begin{pmatrix} 1 & 0 \\ 0 & 0 \end{pmatrix}$, utilizar la matriz densidad para calcular las probabilidades de los distintos resultados que se pueden obtener al medir $\hat{O}$. \textit{(19-01-17)}
    
    \item Un sistema de dos partículas de espín $1/2$, de las que sólo describimos su espín, se encuentra en el estado $\frac{1}{\sqrt{3}} \ket{++} + \frac{i}{\sqrt{3}} \ket{+-} - \frac{1}{\sqrt{3}} \ket{-+}$. Obtener la matriz densidad reducida de la primera partícula (representarla matricialmente). \textit{(19-01-17)}

    \item Un sistema de dos partículas de espín $1/2$, para las cuales sólo describimos el estado de su espín, se encuentra en el estado:
    \[
    \ket{\psi} = \frac{2}{\sqrt{6}} \ket{++} + \frac{i}{\sqrt{6}} \ket{+-} - \frac{1}{\sqrt{6}} \ket{-+}
    \]
    Calcular la matriz densidad reducida de la primera partícula. ¿Corresponde a un estado puro? ¿Por qué? \textit{(15-01-19)}

    \item Tenemos un gran número de partículas que se pueden describir mediante un espacio de estados de dimensión $2$. Si la matriz densidad que describe a las partículas es diagonal, indicar qué forma tendrá si corresponde a un estado puro y a uno mezcla. \textit{(09-02-21)}

    \item Tenemos dos sistemas de espacios de estados de dimensión $2$ y $3$ y de bases $\{\ket{+}, \ket{-}\}$ y $\{\ket{a}, \ket{b}, \ket{c}\}$, respectivamente. Si el estado que describe los dos sistemas es:
    \[
    \ket{\psi} = \frac{1}{\sqrt{10}} \left( \ket{+b} - 2i \ket{-a} + 2 \ket{+c} + i \ket{-c} \right)
    \]
    calcular la matriz densidad reducida del primer sistema. Discutir si corresponde a un estado puro o mezcla. \textit{(19-01-23)}

    \item Dada la siguiente matriz:
    \[
    \hat{\rho} \equiv \begin{pmatrix} 3 & 2i \\ 2i & 1 \end{pmatrix}
    \]
    Indicar de forma razonada si cumple todas las propiedades de una matriz densidad. \textit{(05-09-2023)}

    \item Analizar la siguiente matriz densidad para ver si proviene de un estado puro o mezcla. En caso de provenir de un estado puro encontrar dicho estado:
    \[
    \hat{\rho} \equiv \frac{1}{3} \begin{pmatrix} 1 & i \sqrt{2} \\ -i \sqrt{2} & 2 \end{pmatrix}
    \] \textit{(07-09-2023)}

    \item El estado de espín de dos partículas de espín $1/2$ está descrito por:
    \[
    \ket{\psi} = \frac{1}{\sqrt{3}} \ket{+-} + \frac{1}{\sqrt{3}} \ket{-+} + \frac{i}{\sqrt{3}} \ket{--}
    \]
    Calcular la matriz densidad reducida de la primera partícula. Utilizar dicha matriz densidad para calcular la probabilidad de que la primera partícula tenga $S_z = \hbar/2$. Calcular esta misma probabilidad a partir del estado $\ket{\psi}$ y ver si coinciden. \textit{(11-01-24)}

    \item Desconocemos el estado de una partícula de espín $1/2$, pero conocemos su matriz densidad en la base $\{\ket{\pm}\}$:
    \[
    \hat{\rho} \equiv \frac{1}{8} \begin{pmatrix} 5 & 2 + i \\ 2 - i & 3 \end{pmatrix}
    \]
    ¿Cuál es el valor medio de $S_x$? Cambiar la expresión de $\hat{\rho}$ de la base actual $\{\ket{\pm}\}$ a la base $\{\ket{\pm _x}\}$ y calcular de nuevo la probabilidad anterior.

    Nota.- La matriz de cambio de base es $\hat{C} \equiv \frac{1}{\sqrt{2}} \begin{pmatrix} 1 & 1 \\ 1 & -1 \end{pmatrix}$ \textit{(02-02-24)}

\end{enumerate}

\subsection*{El espín}

\begin{enumerate}
    
    
    \item ¿Para qué se coloca el factor giromagnético de espín 2? \textit{(24-01-14)}
    
    \item Dado el espinor $\pmqty{ \psi_{+} (\va{r}) \\ \psi_{-} (\va{r}) }$, cuál es la densidad de probabilidad de que la partícula se encuentre en la posición $\va{r}$. \textit{(05-09-14)}
    
    \item Escribir la fuerza que sufre un electrón que se mueve en un campo magnético no uniforme de la forma $\va{B} = B(z) \hat{e}_z$. ¿Cómo deducen Ulenbeck y Goudsmit que el electrón debe tener un momento angular intrínseco $s = 1/2$? \textit{(23-01-15)}

    \item Escribir las matrices que representan a los operadores $\hat{S}_x$ y $\hat{S}_y$, los autovalores y los autovectores correspondientes para una partícula de espín $s = 1/2$. \textit{(11-06-15)}

    \item Escribir la expresión de la fuerza que sufre un electrón que se mueve en un campo magnético no uniforme $\va{B} = B(z) \hat{e}_z$. ¿Por qué se introduce el factor giromagnético de espín? \textit{(02-02-16)}


\end{enumerate}

\subsection*{Sistemas de partículas idénticas}

\begin{enumerate}
    
    \item ¿En qué consiste la degeneración de intercambio? \textit{(24-01-14)}

    \item ¿Cómo se definen el Simetrizador y el Antisimetrizador para un sistema de dos partículas? ¿Qué propiedades tienen? \textit{(09-04-14)}

    \item Enunciar el postulado de simetrización. \textit{(05-09-14)}

    \item Enumerar las propiedades del Simetrizador y Antisimetrizador. Explicar brevemente en qué consiste el problema de la degeneración de intercambio. \textit{(15-12-14)}

    \item Tenemos tres fermiones idénticos, de modo que cada uno de ellos puede estar en cuatro estados individuales distintos y ortogonales entre sí: $\ket{A}$, $\ket{B}$, $\ket{C}$ o $\ket{D}$. Calcular cuántos estados distintos podemos formar con estas partículas. Para uno de estos estados calcular la forma del estado utilizando el determinante de Slater. \textit{(11-06-15)}

    \item Demostrar que el operador permutación de un sistema de dos partículas, $\hat{P}_{21}$, es hermítico. \textit{(01-09-16)}

    \item Demostrar que para un sistema formado por partículas idénticas, el carácter de simétrico o antisimétrico se conserva a lo largo de la evolución temporal del sistema. \textit{(02-02-16)}

    \item Demostrar que el operador permutación $\hat{P}_{21}$, para un sistema de dos partículas, es hermítico. \textit{(09-02-17)}

    \item Indicar, de forma razonada, qué afirmaciones son verdaderas y cuáles falsas:
    \begin{itemize}
        \item $\hat{P}_{3214}$ es una transposición.
        \item $\hat{P}_{21}$ es un proyector.
        \item $\hat{P}_{1243} \hat{P}_{3142} = \hat{P}_{3124}$.
        \item $\hat{P}_{213} \ket{1 : u_i; 2 : u_j; 3 : u_k} = \ket{2 : u_i; 1 : u_j; 3 : u_k}$.
    \end{itemize}
    \textit{(08-01-21)}

    \item Tenemos dos partículas idénticas que pueden estar en tres estados individuales distintos: $\ket{A}$, $\ket{B}$ y $\ket{C}$. ¿Cuántos estados distintos hay para el conjunto de las dos partículas: a) si se trata de fermiones idénticos y b) si se trata de bosones idénticos? \textit{(09-02-21)}

    \item Demostrar, basándose en su definición, que el operador permutación de un sistema de dos partículas $\hat{P}_{21}$ es un operador unitario. \textit{(21-01-22)}

    \item Indicar de forma razonada qué afirmaciones son verdaderas y cuáles falsas:
    \begin{itemize}
        \item $\hat{P}_{1423}^{-1} = \hat{P}_{3241}$.
        \item $\hat{P}_{1324}$ es hermítico.
        \item $\hat{P}_{2134}$ es un proyector.
        \item $\hat{P}_{213}$ es una permutación impar.
        \item $(1 + \hat{P}_{132}) / 2$ es un proyector.
    \end{itemize}
    \textit{(11-02-22)}

    \item Indicar y justificar si las siguientes relaciones/afirmaciones son verdaderas o falsas:
    \begin{itemize}
        \item $\hat{P}_{21}^2 = \hat{P}_{21}$
        \item $\hat{P}_{2341} \hat{P}_{3412} = \hat{P}_{4132}$
        \item $\hat{P}_{1243}^\dagger \hat{P}_{1243} = \hat{I}$
        \item $\hat{P}_{321}$ es una permutación par
        \item $\hat{P}_{123} = \hat{I}$
    \end{itemize}
    \textit{(19-01-23)}

    \item Demostrar que el operador $\hat{P}_{21}$ es hermítico. \textit{(07-02-23)}

\end{enumerate}


\subsection*{Entrelazamiento cuántico}

\begin{enumerate}
    
    \item Tenemos un sistema de dos partículas de espín $1/2$, que se encuentran en el estado $\frac{1}{\sqrt{2}} (\ket{+-} + i \ket{-+})$. ¿Es un estado entrelazado? ¿Por qué? Si medimos el valor de $S_x$ de la primera partícula y obtenemos el valor $\hbar/2$, ¿cómo queda el estado de la segunda partícula? \textit{(09-02-17)}

\end{enumerate}
