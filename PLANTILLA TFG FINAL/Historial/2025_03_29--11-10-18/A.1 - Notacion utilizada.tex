\chapter*{Notación utilizada}
\addcontentsline{toc}{chapter}{Notación utilizada}


En este apéndice se recoge la notación utilizada a lo largo de este trabajo. Se ha intentado ser lo más consistente posible, pero en ocasiones se ha tenido que recurrir a notaciones diferentes para evitar confusiones. A continuación se recoge la notación utilizada en este trabajo:

\begin{itemize}
    \item $\va{r}$ : Vector de posición de los electrones.
    \item $\va{R}$ : Vector de posición de los núcleos.
    \item $N$ : Número de electrones del sistema.
    \item $M$ : Número de núcleos del sistema.
    \item $Z$ : Número atómico del átomo.
    \item $V_{\text{ext}}$ : Potencial externo al que están sometidos los electrones.
    \item $n(\va{r})$ : Densidad electrónica.
    \item $n_0$ : Densidad electrónica del estado fundamental.
    \item $\psi_i(\va{r})$ : Funciones de onda de los electrones.
    \item $\phi_I(\va{R})$ : Funciones de onda de los núcleos.
    \item $E_{\text{TF}}[n]$ : Funcional de la energía de Thomas-Fermi.
    \item $F_{\text{ni}}[n(\va{r})]$ : Funcional de la energía cinética de Thomas-Fermi.
    \item $F_{\text{xc}}[n(\va{r})]$ : Funcional de la energía de intercambio y correlación de Thomas-Fermi.
    \item 
\end{itemize}