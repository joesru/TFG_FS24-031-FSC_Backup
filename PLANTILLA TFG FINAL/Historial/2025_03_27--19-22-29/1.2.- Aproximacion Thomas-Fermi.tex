\section{Aproximación de Thomas-Fermi}

Uno de los modelos más simples para la estructura electrónica de los átomos fue desarrollada y presentada en 1927 por Llewellyn Thomas y Enrico Fermi en 1927. Propusieron que la energía cinética del sistema de electrones fuera aproximada como un funcional explícito de la densidad electrónica. Usaron un gas ideal de electrones no interactuantes y homogéneo, de tal manera que la densidad electrónica es igual a la densidad local para cualquier punto.

El funcional en cuestión es de la forma:

\begin{equation}
    F_{\text{ni}} \left[ n \left( \va{r} \right) \right] \approx C_1 \int n^{5/3} \left( \va{r} \right) \dd{\va{r}}
\end{equation}

siendo la constante \begin{equation}
    C_1 = \frac{3}{10} (3 \pi^2)^{2/3} \approx 2.871
\end{equation} en unidades atómicas.

Sin embargo en 1930 Fermi añadiría un termino extra en el que sí tenía en cuenta el intercambio y correlación de los electrones:

\begin{equation}
    F_{\text{xc}}\left[ n \left( \va{r} \right) \right] \approx C_2 \int n^{4/3} \left( \va{r} \right) \dd{\va{r}}
\end{equation}

siendo en este caso \begin{equation}
    C_2 = - \frac{3}{4} \left( \frac{3}{\pi} \right)^{1/3}
\end{equation}

Por lo que considerando tanto el potencial externo al que están sometidos los electrones y la energía electrostática de Hartree, el funcional de la energía nos quedaría de la siguiente manera:

\begin{equation}
    E_{\text{TF}} \left[ n \right] = F_{\text{ni}} + F_{\text{xc}} + \int V_{\text{ext}} \left( \va{r} \right) n \left( \va{r} \right) \dd{\va{r}}  + \frac{1}{2} \int \frac{n \left( \va{r} \right) n \left( \va{r'} \right)}{\abs{ \va{r} - \va{r'} } } \dd{\va{r}} \dd{\va{r'}}
\end{equation}

A su vez todo esto queda restringido por la siguiente ligadura en la que al integrar en todo el espacio la densidad electrónica tenemos como resultado el número de electrones del sistema.

\begin{equation}
    N = \int n \left( \va{r} \right)  \dd{\va{r}}
\end{equation}
