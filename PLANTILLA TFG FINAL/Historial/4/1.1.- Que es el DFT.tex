\section{¿Qué es el DFT?}

\subsection{Definición}

\subsection{Problema con la ecuación de Schrödinger} 

Vamos a ponernos en la situación en la que nos gustaría describir las propiedades de una colección de átomos bien definidas, como podría ser un cristal o una molécula.

Una de las cosas fundamentales que nos gustaría conocer sobre esos átomos es su energías, y sobre todo como cambiaría ésta si movemos los átomos.

Tenemos que tener en cuenta que los núcleos son muchísimo más pesados que los electrones. De esta manera los electrones responden mucho más rápido a cambios en su entorno que los propios núcleos.

Para ello primero resolvemos las ecuaciones que describen el movimiento de los electrones, para posiciones fijas de los núcleos atómicos. Para un conjunto de electrones encontramos su estado fundamental, de manera que por la aproximación de Born-Oppenheimer podemos considerar como problemas matemáticos distintos el de los electrones y el del núcleo.