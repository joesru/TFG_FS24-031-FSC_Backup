\section{¿Qué es el DFT?}

\subsection{Definición}

\subsection{Problema con la ecuación de Schrödinger} 

Vamos a ponernos en la situación en la que nos gustaría describir las propiedades de una colección de átomos bien definidas, como podría ser un cristal o una molécula.

Una de las cosas fundamentales que nos gustaría conocer sobre esos átomos es su energía, y sobre todo como cambiaría ésta si movemos los átomos.

Tenemos que tener en cuenta que los núcleos son muchísimo más pesados que los electrones. De esta manera los electrones responden mucho más rápido a cambios en su entorno que los propios núcleos.

Para ello primero resolvemos las ecuaciones que describen el movimiento de los electrones, con posiciones fijas de los núcleos atómicos. Encontraremos el estado fundamental para un conjunto de electrones, de manera que por la aproximación de Born-Oppenheimer podemos considerar como problemas matemáticos distintos el de los electrones y el del núcleo.

Si tenemos $M$ núcleos en las posiciones $\va{R}_1 , \dots , \va{R}_M$, podemos expresar la energía fundamental del estado como función de la posición de estos núcleos: $E = E(\va{R}_1 , \dots , \va{R}_M)$. Conociendo esta energía como la energía potencial adiabática de la superficie de los átomos.

Por otra parte para los $N$ electrones que conforman el sistema, la ecuación a resolver es la de Schrödinger: 

\begin{equation}
    \left[ \frac{\hbar^2}{2m} \sum_{i=1}^{N} \laplacian_i + \sum_{i=1}^{N} V(\va{r}_i) + \sum_{i=1}^{N} \sum_{j<i} U(\va{r}_i , \va{r}_j) \right] \psi = E \psi
\end{equation}


Para el hamiltoniano que hemos elegido, $\psi$ es la función de onda electrónica que es función de las coordenadas espaciales de cada uno de los $N$ electrones, $\psi = \psi \left( \va{r}_{1} , \dots , \va{r}_{N} \right)$  y $E$ es la energía del estado fundamental de los electrones.

Debido a las propiedades de los electrones la función de onda la podemos poner como producto de cada una de las funciones de onda individuales de cada uno de los N electrones, $\psi = \psi_{1} \left( \va{r} \right) \cdots \psi_{N} \left( \va{r} \right)$ 

Es importante saber que $N \gg M$, ya que por cada núcleo hay varios electrones asociados. Esto nos da cuenta de que resolver la ecuación de Schrödinger para materiales prácticos es muy complicado por la cantidad de dimensiones y ecuaciones a resolver. 

La situación se complica aún más cuando miramos el hamiltoniano, siendo el término más crítico el de interacción entre electrones desde el punto de vista de resolver la ecuación. Para un electrón dado no es simple calcular $\psi_{i} \left( \va{r} \right)$ ya que depende del resto de electrones del sistema.

Una cosa que si que podemos medir de forma mas sencilla es la probabilidad de que los $N$ estén en una configuración particular de coordenadas $\va{r}_{1} , \dots , \va{r}_{N}$. De tal manera que una cantidad que está relacionada con esta probabilidad es la densidad de electrones en una determinada posición del espacio, 

\begin{equation}
    n(\va{r}) = 2 \sum_i \psi^*_i (\va{r}) \psi_i (\va{r})
\end{equation}

, expresión en la cual ya se tiene en cuenta el Principio de exclusión de Pauli con el 2 que precede a la suma.

Lo importante de esta discusión es que la densidad electrónica, $n(\va{r})$, es un parámetro del sistema que sí podemos medir físicamente y de la que podemos obtener gran parte de la información del sistema al estar relacionada con la función de onda total  $\psi = \psi \left( \va{r}_{1} , \dots , \va{r}_{N} \right)$.